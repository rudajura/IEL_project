\section{Příklad 1}
% Jako parametr zadejte skupinu (A-H)
\prvniZadani{E} 

\subsection{Zjednodušení obvodu na \R{ekv}}
\begin{figure}[H]
Sečtení sériově zapojených napěťových zdrojů: $$U = \Uu{1} + \Uu{2} = (115 + 55) = \SI{170}{\volt}$$ a zjednodušení rezistorů \R{3} a \R{4} na: $$\R{45} = \frac{\R{4}\times\R{5}}{\R{4}+\R{5}} => \R{45} = \frac{340\times575}{340+575} = \SI{213.6612}{\ohm}$$

    \centering
    \begin{circuitikz}
    \draw
    (0,0) to[dcvsource, v^<=U] (0,4)
    -- (0,4) -- (1,4)
    (1,4) node[circ]{} (1,4)
    (1,4) to[R, l^=\R{1}] (3,4)
    (3,4) node[circ]{} (3,4)
    (3,4) to[R, l^=\R{3}] (3,2)
    (3,2) to[R, l^=\R{2}] (1,2)
    -- (1,2) -- (1,4)
    (3,4) to[R, l^=\R{45}] (5,4)
    (5,4) to[R, l^=\R{7}] (5,2)
    (5,2) to[R, l^=\R{6}] (3,2)
    (3,2) node[circ]{} (3,2)
    (5,2) node[circ]{} (5,2)
    -- (5,2) -- (5,0)
    (5,0) to[R, l^=\R{8}] (0,0)
    {[anchor=south east] (1,4) node {A} (3,4) node {B} (3,2) node {C}};
    \end{circuitikz}
\end{figure}

\begin{figure}[H]
Transfigurace trojúhelník $\rightarrow$ hvězda.

    \centering
    \begin{circuitikz}
    \draw
    (0,0) to[dcvsource, v^<=U] (0,4)
    (0,4) node[circ]{} (0,4)
    (0,4) to[R, l^=\R{A}] (2,4)
    (2,4) to[R, l^=\R{B}] (4,5)
    (2,4) to[R, l^=\R{C}] (4,3)
    (4,5) node[circ]{} (4,5)
    (4,3) node[circ]{} (4,3)
    (4,5) to[R, l^=\R{45}] (6,5)
    (6,5) to[R, l^=\R{7}] (6,3)
    (4,3) to[R, l^=\R{6}] (6,3)
    -- (6,3) -- (6,0)
    (6,0) to[R, l^=\R{8}] (0,0)
    {[anchor=south] (0,4) node {A} (4,5) node {B} (4,3) node {C}};
    \end{circuitikz} 
\end{figure}

\begin{figure}
    \begin{align*}
        %díky align* se napisou rovnice ve stejne velikosti a nezmensi se,
        %jako v pripade {figure}
        \R{A}& = \frac{\R{1}\times\R{2}}{\R{1} + \R{2} + \R{3}} = \frac{485\times660}{485 + 660 + 100} = \SI{257.1084}{\ohm} \\
    	\R{B}& = \frac{\R{1}\times\R{3}}{\R{1} + \R{2} + \R{3}} = \frac{485\times100}{485 + 660 + 100} = \SI{38.9558}{\ohm} \\
    	\R{C}& = \frac{\R{2}\times\R{3}}{\R{1} + \R{2} + \R{3}} = \frac{660\times100}{485 + 660 + 100} = \SI{53.012}{\ohm} \\
    \end{align*}
\end{figure}

\begin{figure}[H]
Zjednodušení rezistorů \R{B}, \R{45} a \R{7} v sérii do jednoho: $$\R{B457} = \R{B} + \R{45} + \R{7} = (38.9558 + 213.6612 + 255) = \SI{507.617}{\ohm} $$\\
a rezistorů \R{C} a \R{6}: $$\R{C6} = \R{C} + \R{6} = (53.012 + 815) = \SI{868.012}{\ohm}$$
    
    \centering
    \begin{circuitikz}
    \draw
    (0,0) to[dcvsource, v^<=U] (0,3)
    (0,3) to[R, l^=\R{A}] (2,3)
    (2,3) node[circ]{} (2,3)
    -- (2,2) -- (2,4)
    (2,4) to[R, l^=\R{B457}] (4,4)
    (2,2) to[R, l^=\R{C6}] (4,2)
    -- (4,4) -- (4,0)
    (4,2) node[circ]{} (4,2)
    (0,0) to[R, l^=\R{8}] (4,0);
    \end{circuitikz}
\end{figure}

\begin{figure}[H]
Zjednodušení rezistorů \R{B457} a \R{C6} zapojených paralelně do $$\R{B457C6} = \frac{\R{B457}\times\R{C6}}{\R{B457} + \R{C6}} = \frac{507.617\times868.012}{507.617 + 868.012} = \SI{320.3027}{\ohm}$$ \\

    \centering
    \begin{circuitikz}
    \draw
    (0,0) to[dcvsource, v^<=U] (0,2)
    (0,2) to[R, l^=\R{A}] (2,2)
    (2,2) to[R, l^=\R{B457C6}] (4,2)
    (4,2) to[R, l^=\R{8}] (6,2)
    -- (6,2) -- (6,0)
    -- (6,0) -- (0,0);
    \end{circuitikz}
\end{figure}

\begin{figure}[H]
Konečný výpočet \R{EKV} (sériově zapojené rezistory \R{A}, \R{B457C6} a \R{8}) $$\R{EKV} = \R{A} + \R{B457C6} + \R{8} = (257.1084 + 320.3027 + 225) = \SI{802.4111}{\ohm}$$

    \centering
    \begin{circuitikz}
    \draw
    (0,0) to[dcvsource, v^<=U] (0,2)
    (0,2) to[R, l^=\R{EKV}] (2,2)
    -- (2,2) -- (2,0)
    -- (2,0) -- (0,0);
    \end{circuitikz}
\end{figure}

\subsection{Výpočet celkového proudu a dosazení}
\begin{figure}[H]
Celkový proud vypočítáme Ohmovým zákonem dosazením \R{EKV} do rovnice: $$I = \frac{\Uu{12}}{\R{EKV}} = \frac{170}{802.4111} = \SI{0.2119}{\ampere}$$
\end{figure}

\begin{figure}[H]
Dále potřebujeme vypočítat napětí na \R{B357C6}, abychom následně mohli vypočítat proud \I{R7}:
$$\Uu{RA} = \R{A}\times\I = 257.1084\times0.2119 = \SI{54.4813}{\volt}$$
$$\Uu{R8} = \R{8}\times\I = 225\times0.2119 = \SI{47.6775}{\volt}$$
$$\Uu{RB457C6} = \Uu{12} - \Uu{RA} - \Uu{R8} = (170 - 54.4813 - 47.6775) = \SI{67.8412}{\volt}$$

    \centering
    \begin{circuitikz}
    \draw
    (0,0) to[dcvsource, v^<=U] (0,2)
    (0,2) to[R, l^=\R{A}] (2,2)
    (2,2) to[R, l^=\R{B457C6}] (4,2)
    (4,2) to[R, l^=\R{8}] (6,2)
    -- (6,2) -- (6,0)
    -- (6,0) -- (0,0);
    \end{circuitikz}
\end{figure}

\begin{figure}[H]
Výpočet \I{R7} za pomoci Ohmova zákona a II. Kirchhoffova zákona: 
$$\I{R7} = \frac{\Uu{RB457C6}}{\R{B457}} = \frac{67.8412}{507.617} = \SI{0.1336}{\ampere}$$
a následně \Uu{R7} dosazením:
$$\Uu{R7} = \R{7}\times\I{R7} = 255\times0.1336 = \SI{34.068}{\volt}$$

    \centering
    \begin{circuitikz}
    \draw
    (0,0) to[dcvsource, v^<=U] (0,3)
    (0,3) to[R, l^=\R{A}] (2,3)
    (2,3) node[circ]{} (2,3)
    -- (2,2) -- (2,4)
    (2,4) to[R, l^=\R{B457}, i=\I{R7}] (4,4)
    (2,2) to[R, l^=\R{C6}] (4,2)
    -- (4,4) -- (4,0)
    (4,2) node[circ]{} (4,2)
    (0,0) to[R, l^=\R{8}] (4,0);
    \end{circuitikz}

P.S. Napětí \Uu{R7} mi ve Fastladu vyšlo \SI{34.089}{\volt}, nejspíše je to dáno tím, že jsem zaokrouhloval mezivýsledky, avšak zaokrouhloval jsem je na 4 desetinná místa, jak bylo v instrukcích.
\end{figure}
