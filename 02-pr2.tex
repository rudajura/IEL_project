\section{Příklad 2}
% Jako parametr zadejte skupinu (A-H)
\druhyZadani{H}

\subsection{Zjednodušení obvodu pomocí Théveninova teorému}
\begin{figure}[H]
Napěťový zdroj nahradíme zkratem a rezistor \R{1} odpojíme.
\end{figure}

\begin{figure}[H]
    \centering
    \begin{circuitikz}
    \draw
    (0,0) -- (0,2)
    (0,2) to[R, l^=\R{2}] (2,2)
    (2,2) to[R, l^=\R{4}] (4,2)
    (2,2) to[R, l^=\R{3}] (2,0)
    (2,0) to[R, l^=\R{5}] (4,0)
    (0,3) -- (0,2)
    (2,3) node[ocirc]
    (2,3) -- (2,2)
    (4,2) -- (4,0)
    (0,0) -- (2,0)
    (0,3) node[ocirc];
    \end{circuitikz}
\end{figure}

\begin{figure}[H]
Převedení sériově zapojených rezistorů \R{4} a \R{5} do jednoho: $$\R{45} = \R{4} + \R{5} = (205 + 560) = \SI{765}{\ohm}$$
a následně paralelně zapojených rezistorů \R{45} a \R{3} do jednoho: $$\R{453} = \frac{\R{45}\times\R{3}}{\R{45} + \R{3}} = \frac{765\times580}{765 + 580} = \SI{329.8885}{\ohm}$$
a přípojíme voltmetr \Uu{i} a napěťový zdroj U
\end{figure}

\begin{figure}[H]
    \centering
    \begin{circuitikz}
    \draw
    (0,0) to[dcvsource, v^<=U] (0,2)
    (0,4) node[ocirc]
    (0,4) -- (0,2)
    (0,2) to[R, l^=\R{2}] (2,2)
    (2,2) to[R, l^=\R{453}] (4,2)
    (4,2) -- (4,0)
    (4,0) -- (0,0)
    (2,4) node[ocirc]
    (2,4) -- (2,2)
    (0,4) to[dcvsource, v^=\Uu{i}] (2,4);
    \end{circuitikz}
\end{figure}

\begin{figure}[H]
Spočítáme proud \Uu{i} z rovnice $$\Uu{i} = U\times\frac{\R{2}}{\R{2}+ \R{453}} = 220\times\frac{360}{360 + 329.8885} = \SI{114.8012}{\volt}$$
\end{figure}

\begin{figure}[H]
Zjednodušení rezistorů \R{2} a \R{453} do jednoho: $$\R{i} = \frac{\R{453}\times\R{2}}{\R{453} + \R{2}} = \frac{329.8885\times360}{329.8885 + 360} = \SI{172.1436}{\ohm}$$
a připojení rezistoru \R{1}
    
    \centering
    \begin{circuitikz}
    \draw
    (0,0) to[dcvsource, v^<=\Uu{i}] (0,2)
    (0,2) to[R, l^=\R{i}] (2,2)
    (2,2) node[ocirc]
    (2,2) -- (3,2)
    (3,2) to[R, l^=\R{1}] (3,0)
    (0,0) -- (3,0)
    (2,0) node[ocirc];
    \end{circuitikz}
\end{figure}

\subsection{Dosazení a výpočet \R{1} a \I{1}}
\begin{figure}[H]
$$\I{R1} = \frac{\Uu{i}}{\R{i} + \R{1}} = \frac{114.8012}{172.1436 + 190} = \SI{0.317}{\ampere}$$
$$\Uu{R1} = \R{1}\times\I{R1} = 190\times0.317 = \SI{60.23}{\volt}$$
\end{figure}