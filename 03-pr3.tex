\section{Příklad 3}
% Jako parametr zadejte skupinu (A-H)
\tretiZadani{E}

\subsection{Nahrazení napěťového zdroje za proudový zdroj}
Nahradíme si napěťový zdroj za proudový \I{3}, pro který platí: $$\I{3} = \Gg{5}\times\Uu{} = \frac{21}{135} = \SI{0.1556}{\ampere}$$
$$G = \frac{1}{R}$$

\begin{figure} [H]
    \centering
    \begin{circuitikz}
    \draw
    (0,0) to[ioosource, i=\I{1}] (0,2)
    (0,2) -- (2,2)
    (2,2) to[R, l^=\Gg{4}] (4,2)
    (2,2) -- (2,4)
    (2,3) to[R, l^=\Gg{5}] (4,3)
    (4,4) to[ioosource, i=\I{3}] (2,4)
    (4,4) -- (4,2)
    (4,2) to[R, l^=\Gg{3}] (4,0)
    (2,2) to[R, l^=\Gg{1}] (2,0)
    (4,2) -- (6,2)
    (6,0) to[ioosource, i=\I{2}] (6,2)
    (6,0) -- (4,0)
    (4,0) to[R, l^=\Gg{2}] (2,0)
    (0,0) -- (2,0)
    (2,2) node[circ]
    
    (2,0) node[circ]
    
    (4,0) node[circ]
    
    (4,2) node[circ]
    
    (4,3) node[circ]
    
    (2,3) node[circ]
    
    {[anchor=south east] (2,2) node {A}}
    {[anchor=south west] (4,2) node {B}}
    {[anchor=north] (4,0) node {C}};
    \end{circuitikz}
\end{figure}

\begin{figure}[H]
Vyjádříme si proudy: \\
$$\I{R1} = \Gg{1}\times\Uu{A}$$
$$\I{R2} = \Gg{2}\times\Uu{C}$$
$$\I{R3} = \Gg{3}\times(\Uu{B} - \Uu{C})$$
$$\I{R4} = \Gg{4}\times(\Uu{A} - \Uu{B})$$
$$\I{R3} = \Gg{5}\times(\Uu{A} - \Uu{B})$$

Nyní vytvoříme rovnice pro uzly (proudy, které protékají uzly A, B a C):
$$A: \I{1} + \I{R5} - \I{R4} - \I{R1} = 0$$
$$B: \I{2} - \I{R3} + \I{R4} = 0$$
$$C: \I{R3} - \I{R2} - \I{2} = 0$$
\end{figure}

\begin{figure}
Proudy rozložíme na součin odporů a uzlových napětí:
$$\Gg{5}(\Uu{A} - \Uu{B}) - \Gg{4}(\Uu{A} - \Uu{B}) - \Gg{1}\Uu{A} = -\I{1}$$
$$-\Gg{3}(\Uu{B} - \Uu{C}) + \Gg{4}(\Uu{A} - \Uu{B}) = -\I{A}$$
$$\Gg{3}(\Uu{B} - \Uu{C}) - \Gg{2}\Uu{C} = \I{2}$$

Vytkneme si jednotlivá napětí:
$$\Uu{A}(\Gg{5} - \Gg{4} - \Gg{1}) + \Uu{B}(-\Gg{5} + \Gg{4}) = -\I{1}$$
$$ \Uu{A}(\Gg{4}) + \Uu{B}(-\Gg{3} - \Gg{4}) + \Uu{C}(\Gg{3}) = -\I{2}$$
$$\Uu{B}(\Gg{3}) +  \Uu{C}(-\Gg{3} - \Gg{2}) = \I{2}$$
\end{figure}

\begin{figure}[H]
Soustavu rovnic převedeme do matic ve formě $Ax = B$ \\
A je matice odporů, x je matice napětí a B je matice výsledných proudů
\begin{align*}
    \begin{pmatrix}
    \Gg{5}-\Gg{4}-\Gg{1}&-\Gg{5}+\Gg{4}&0 \\
    \Gg{4}&-\Gg{3}-\Gg{4}&\Gg{3} \\
    0&\Gg{3}&-\Gg{3}-\Gg{2}
    \end{pmatrix}\times
    \begin{pmatrix}
    \Uu{A} \\
    \Uu{B} \\
    \Uu{A}
    \end{pmatrix} = 
    \begin{pmatrix}
    -\I{1} \\
    -\I{2} \\
    \I{2}
    \end{pmatrix}
\end{align*}
\subsection{Počítání matic a determinantů}
Dosazení do matice:
\begin{align*}
    \begin{pmatrix}
    \frac{5}{1092}&-\frac{1}{42}&0 \\[5pt]
    \frac{1}{42}&-\frac{47}{1092}&\frac{1}{52} \\[5pt]
    0&\frac{1}{52}&-\frac{47}{1092}
    \end{pmatrix}\times
    \begin{pmatrix}
    \Uu{A} \\[5pt]
    \Uu{B} \\[5pt]
    \Uu{A}
    \end{pmatrix} = 
    \begin{pmatrix}
    -\frac{11}{20} \\[5pt]
    -\frac{13}{20} \\[5pt]
    \frac{13}{20}
    \end{pmatrix}
\end{align*}
\end{figure}

\begin{figure}[H]
Vypočítáme determinanty $\Delta, \Delta_B, \Delta_C$ pomocí Sarrusova pravidla:
\begin{gather*}
    \Delta = 
    \begin{vmatrix}
    \frac{5}{1092}&-\frac{1}{42}&0 \\[5pt]
    \frac{1}{42}&-\frac{47}{1092}&\frac{1}{52} \\[5pt]
    0&\frac{1}{52}&-\frac{47}{1092}
    \end{vmatrix} = 
    (\frac{11045}{1302170688} - \frac{5}{2952768} - \frac{47}{1926288}) = -\frac{1}{56784}
\end{gather*}
\begin{gather*}
    \Delta_B = 
    \begin{vmatrix}
    \frac{5}{1092}&-\frac{11}{20}&0 \\[5pt]
    \frac{1}{42}&-\frac{13}{20}&\frac{1}{52} \\[5pt]
    0&\frac{13}{20}&-\frac{47}{1092} 
    \end{vmatrix} = 
    (\frac{3055}{23849280} - \frac{65}{1135680} - \frac{517}{917280}) = -\frac{113}{229320}
\end{gather*}
\begin{gather*}
    \Delta_C = 
    \begin{vmatrix}
    \frac{5}{1092}&-\frac{1}{42}&-\frac{11}{20} \\[5pt]
    \frac{1}{42}&-\frac{47}{1092}&-\frac{13}{20} \\[5pt]
    0&\frac{1}{52}&-\frac{13}{20} 
    \end{vmatrix} = 
    (-\frac{3055}{23849280} - \frac{11}{43680} - \frac{65}{1135680} + \frac{13}{35280}) = \frac{1}{21804}
\end{gather*}
\end{figure}

\subsection{Dosazení}
\begin{figure}[H]
Využitím Cramerova pravidla vypočítáme uzlová napětí:
$$\Uu{B} = \frac{\Delta_B}{\Delta} = \frac{-\frac{113}{229320}}{-\frac{1}{56784}} = \frac{2938}{105} = \SI{27.981}{\volt}$$
$$\Uu{C} = \Uu{R2} = \frac{\Delta_C}{\Delta} = \frac{\frac{1}{21840}}{-\frac{1}{56784}} = -\frac{13}{5} = \SI{2.6}{\volt}$$

Vypočítáme napětí:
$$\Uu{R3} = \Uu{B} - \Uu{C} = (\frac{2938}{105} + \frac{13}{5}) = \SI{30.581}{\volt}$$
a proud
$$\I{R3} = \frac{\Uu{R3}}{\R{3}} = \frac{\frac{3211}{105}}{52} = \SI{0.5881}{\ampere}$$
\end{figure}