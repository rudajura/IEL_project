\section{Příklad 4}
% Jako parametr zadejte skupinu (A-H)
\ctvrtyZadani{E}

\subsection{Sestavení a vyjádření smyčkových proudů}

\begin{figure}[H]
    \centering
    \begin{circuitikz}
    \ctikzset{cute inductors}
    \draw
    (0,4) to[L, l=\Ll{1}] (2,4)
    (2,4) to[sV = \Uu{1}] (5,4)
    (0,4) -- (0,0)
    (0,2) node[circ] {}
    (0,2) to[R, l=\R{1}] (2,2)
    (2,2) to[C, l=\Cc{2}] (5,2)
    (5,2) node[circ] {}
    (2,2) node[circ] {}
    (2,2) to[sV=\Uu{2}] (2,0)
    (0,0) to[C, l=\Cc{1}] (2,0)
    (2,0) to[R, l=\R{2}] (5,0)
    (2,0) node[circ] {}
    (5,2) to[L, l=\Ll{2}] (5,0)
    (5,2) node[circ] {}
    (5,4) -- (5,2)
    (3,3) node[circulator, scale = 0.6] {}
    (1,1.1) node[circulator, scale = 0.6] {}
    (4,1.1) node[circulator, scale = 0.6] {}
    {[anchor=south west] (3,3) node{\I{A}} (1,1.1) node{\I{B}} (4,1.1) node{\I{C}}};
    \end{circuitikz}
\end{figure}
\begin{figure}[H]
Kondenzátory a cívky si převedeme na impedance pro snažší manipulaci s nimi pro počítání v matici.
$$\omega = 2\pi f => 2\pi \times 180 => 180\pi \; \text{rad/s}$$
$$Z_L_1 = j\omega L_1 = 180\pi \times 0.13 = 73.5133j\omega$$
$$Z_L_2 = j\omega L_2 = 180\pi \times 0.06 = 33.9292j\omega$$
$$Z_C_1 = -\frac{j}{\omega C_1} = -\frac{j}{180\pi \times 100\times 10^{-6}} = -17.6839j\omega$$
$$Z_C_2 = -\frac{j}{\omega C_2} = -\frac{j}{180\pi \times 65\times 10^{-6}} = -27.206j\omega$$
\end{figure}
\begin{figure}[H]
Vyjádříme si smyčkové proudy:
$$I_A: Z_L_1I_A + Z_C_2(I_A - I_C) + R_1(I_A - I_B) = 0$$
$$I_B: R_1(I_B - I_A) + U_2 + Z_C_1I_B = 0$$
$$I_C: (R_2 + Z_L_2)I_C + Z_C_2(I_C - I_A) - U_2 = 0$$

Dále vytvoříme matici pro smyčkové proudy:
\begin{align*}
    \begin{pmatrix}
    Z_L_1 + Z_C_2 + R_1&-R_1&-Z_C_2 \\
    -R_1&Z_C_1 + R_1&0 \\
    -Z_C_2&0&R_2 + Z_L_2 + Z_C_2
    \end{pmatrix} \times
    \begin{pmatrix}
    I_A \\
    I_B \\
    I_C
    \end{pmatrix} = 
    \begin{pmatrix}
    -U_1 \\
    -U_2 \\
    U_2
    \end{pmatrix}
\end{align*}
\end{figure}

\begin{figure}[H]
Dosadíme hodnoty do matice:
\begin{align*}
    \begin{pmatrix}
    14+46.3073j&-14&27.206j \\
    -14&14 - 17.6839j&0 \\
    27.206j&0&13 + 6.7232j
    \end{pmatrix} \times
    \begin{pmatrix}
    I_A \\
    I_B \\
    I_C
    \end{pmatrix} = 
    \begin{pmatrix}
    -5 \\
    -3 \\
    3
    \end{pmatrix}
\end{align*}
\end{figure}

\subsection{Výpočet smyčkových proudů}
Matice spočítáme pomocí Cramerova pravidla:
\begin{align*}
    \Delta =  
    \begin{vmatrix}
    14+46.3073j&-14&27.206j \\
    -14&14 - 17.6839j&0 \\
    27.206j&0&13 + 6.7232j
    \end{vmatrix} = 
    18313.7764 - 2373.9837j
\end{align*}

\begin{align*}
    \Delta_C =  
    \begin{vmatrix}
    14+46.3073j&-14&-5 \\
    -14&14 - 17.6839j&-3 \\
    27.206j&0&3
    \end{vmatrix} = 
    4862.2219 + 4249.2548j
\end{align*}
$$I_C = I_L_2 = \frac{\Delta_C}{\Delta} = \frac{4862.2219 + 4249.2548j}{18313.7764 - 2373.9837j} = (0.2315 + 0.262j)\Am$$
Vypočítáme napětí $U_{L2}$:
$$U_{L2} = I_{L2}\times Z_{L2} = (0.2315 + 0.262j)\times 33.9292j = (-8.8895 + 7.8546j)\Vo$$
Vypočítáme $|U_{L2}|$ a $\varphi_{L2}$:
\begin{gather*}
|U_{L2}| = \sqrt{{Re}{(U_{L2})^2} + {Im}{(U_{L2})^2}} = \sqrt{(-8.8895)^2 + (7.8546)^2} = \SI{11.8625}{\volt} \\
\varphi_{L2} = arctg(\frac{Im(U_{L2})}{Re(U_{L2})} = arctg\frac{7.8546}{-8.8895} = -0.7237 rad \\
\end{gather*}
Protože jsme ve II. kvadrantu, musíme připočíst $\pi$:
$$\varphi_{L2} = \pi - 0.7237 = 2.4179 rad = 138.5355^\circ$$